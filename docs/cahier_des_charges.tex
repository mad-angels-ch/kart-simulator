\documentclass{article}

\usepackage{hyperref}

\title{Kart simulator - cahier des charges}
\date{2021-08-26}
\author{Lorin Jacot, Noé Thiran}

\begin{document}
\maketitle

\section{Version intermédiaire}

Un circuit de Kart jouable pour un joueur.
\begin{itemize}
    \item La page d'accueil du jeu est un menu avec les options suivantes: \begin{itemize}
              \item Quitter le jeu
              \item Règles du jeu qui sont les suivantes: \begin{itemize}
                        \item Le but est de compléter le nombre de tours requis le plus rapidement possible (avant les autres joueurs pour la version finale)
                        \item Les karts perdent de la vitesse lorsqu'ils touchent le bord de la piste ou des obstacles quelconques
                        \item La partie est perdue si le joueur sort de la piste (par un bout sans barrières)
                    \end{itemize}
              \item Afficher les contrôles: \begin{itemize}
                        \item W, A, S, D sont les touches directionelles
                        \item Esc permet de mettre le jeu en pause et d'afficher le menu ingame
                    \end{itemize}
              \item Options de customisation suivantes: \begin{itemize}
                        \item Choix de la musique éventuelle
                        \item Option pour désactiver le son
                    \end{itemize}
              \item Lancer une partie: \begin{itemize}
                        \item Une liste à dérouler permet de choisir le circuit
                        \item Le menu principal disparaît et le circuit sélectionné est affiché à la place
                        \item Un compte à rebours au centre de l'écran se lance et indique le début de la course
                    \end{itemize}
          \end{itemize}
    \item Déroulement de la course: \begin{itemize}
              \item Lorsque le jeu est mis sur pause: \begin{itemize}
                        \item Le temps s'arrête et le kart se fige instantanément
                        \item Un menu s'affiche avec les options suivantes: \begin{itemize}
                                  \item Menu d'options, identique à celui du menu principal
                                  \item Quitter la partie
                                  \item Reprendre la partie
                              \end{itemize}
                    \end{itemize}
              \item Le jeu se termine lorsque: \begin{itemize}
                        \item Lorsque le kart franchi la ligne d'arrivée à la fin du dernier tour
                        \item Lorsque le kart sort de la piste.
                        \item Le joueur quitte la partie.
                    \end{itemize}
              \item Physique du jeu: \begin{itemize}
                        \item Les karts rebondissent contre les bords de piste, les autres karts et les autres obstacles
                        \item Le kart disparaît de l'écran s'il sort de la piste
                    \end{itemize}
              \item POV: \begin{itemize}
                        \item L'ensemble de la piste se trouve sur l'écran
                        \item La piste est fixe et c'est le kart qui tourne et bouge
                    \end{itemize}
          \end{itemize}
    \item Technologies utilisées: \begin{itemize}
              \item Interface graphique: \url{https://kivy.org/#home}
          \end{itemize}
\end{itemize}

\section{Version finale}

Le jeu peut être joué en multi-joueurs.
\begin{itemize}
    \item Ajout d'une database contenant les informations suivantes: \begin{itemize}
              \item Utilisateurs
              \item Les différents circuits
              \item Les données des parties jouées (les scores)
          \end{itemize}
    \item Modifications de la page d'accueil: \begin{itemize}
              \item Ajout des options suivantes: \begin{itemize}
                        \item S'identifier ou créer un compte avec les fonctionnalités suivantes: \begin{itemize}
                                  \item Comptes enregistrés sur le cloud (possibilité de s'identifier depuis n'importe quel ordinateur)
                                  \item Accéder à sa liste de scores et comparer ceux-ci aux meilleurs scores (d'autres joueurs) (le fonctionnement des scores reste à déterminer)
                                  \item Choix du pseudo
                              \end{itemize}
                        \item Choix du mode de jeu: \begin{itemize}
                                  \item Solo (identique à la version intermédiaire)
                                  \item Créer une partie multi-joueurs (un lien/numéro de partie est créé, permettant à d'autres joueurs de la rejoindre). Il pourra, à l'aide d'un boutton dédié, démarrer le compte à rebours du départ de la course, au moment qu'il estime le plus opportun.
                                  \item Rejoindre une partie multi-joueurs
                              \end{itemize}
                    \end{itemize}
          \end{itemize}
    \item Modifications du déroulement de la course: \begin{itemize}
              \item Mode "pause": \begin{itemize}
                        \item Le menu apparaît toujours avec les mêmes options
                        \item Le jeu n'est plus mis en pause: la course continue
                        \item L'option "Quitter la partie" déconnecte le joueur et l'amène au menu principal
                    \end{itemize}
              \item Le jeu se termine lorsque tous les karts ont franchi la ligne d'arrivée. Si un joueur ayant terminé la course décide de ne pas attendre les autres joueur pour quitter la partie, son score est tout de même comptabilisé
              \item Affichage du pseudo du joueur à côté du kart
          \end{itemize}
    \item Technologies utilisées: \begin{itemize}
              \item Interface graphique: \href{https://kivy.org/#home}{Kivy}
              \item Communication serveur-client: \href{https://python-socketio.readthedocs.io/en/latest/index.html}{Python-socketio}
              \item Webframework: \href{https://flask.palletsprojects.com/en/2.0.x/}{Flask}
          \end{itemize}

\end{itemize}

\section{Extensions possibles}
\begin{itemize}
    \item Création de circuits :
          \begin{itemize}
              \item Basé sur le \href{https://lj44.ch/creator/flipper}{créateur de monde du flipper} avec les modification suivantes: \begin{itemize}
                        \item Choix du tracé
                        \item Choix des dimensions
                        \item Choix du fond
                        \item Choix de la musique
                        \item Choix du nombre de tours
                        \item Support d'images (en plus des couleurs) pour les objets ajoutés
                    \end{itemize}
          \end{itemize}
    \item Amélioration de la POV de la version intermédiaire: zoom sur le circuit dynamique et relatif à la position du kart (du type \href{https://youtu.be/ngyf3bssoKY}{Brawl Stars})
    \item Minimap
    \item Possibilité de récompenses en cas de bonne performance: pièces de kart (aileron, roues, skin, ...)
    \item Kart: \begin{itemize}
              \item Choix du modèle
              \item Possibilité de dérapage
              \item Customisation (aileron, roues, skin, ...)
          \end{itemize}
    \item Possibilité de se tirer dessus, les karts meurent lorsqu'ils n'ont plus de points de vie
    \item Mondes sur plusieurs niveaux (fausse 3D) (ex: ponts, tunnels)
    \item Choix du pilote, avec différentes exclamations de ceux-ci durant la course
    \item Nouveaux types d'obstacles, par exemple des obstacles ralentissants ou accélérants
    \item Choix de la POV \begin{enumerate}
              \item Identique à celle de la version intermédiaire
              \item Le kart est immobile et au centre de l'écran, c'est le circuit qui bouge et tourne
          \end{enumerate}
    \item ...
\end{itemize}

% \section{Extensions impossibles}
% \begin{itemize}
%     \item IA: \begin{itemize}
%         \item Karts gérés par l'ordinateur
%         \item Commentateur dynamique
%         \item Calcul + démonstration du meilleur tracé
%     \end{itemize}
% \end{itemize}

\end{document}