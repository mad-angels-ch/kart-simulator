\documentclass{article}

\usepackage{hyperref}

\title{Kart simulator - cahier des charges}
\date{2021-08-26}
\author{Lorin Jacot, Noé Thiran}

\begin{document}
\maketitle
\pagenumbering{gobble}

Le karting est une discipline du sport automobile qui se pratique sur des circuits de 700 à 1 500 mètres environ et d'une largeur de 8 mètres en moyenne pouvant accueillir jusqu'à une quarantaine de karts simultanément lors d'une course. Les karts (appelés aussi "go-karts" en Belgique et au Québec) sont de petites voitures monoplaces à quatre roues, équipées d’un moteur de petite cylindrée (en général des deux-temps de 100 ou 125 cm3) pouvant développer jusqu'à plus de 40 chevaux, pour un poids inférieur à 175 kg avec le pilote, ce qui en fait des engins de course très performants.

Les karts de 125 cm3 sans boîte de vitesses développent de 15 à 35 chevaux environ.
Les karts de 125 cm3 à boîte de vitesses peuvent atteindre 185 km/h sur circuit long (type Carole, en France)1 et effectuer le 0 à 100 km/h en un peu plus de 3 secondes.
Les 250 cm3, nommés aussi "Superkart", sont des bolides extrêmement performants qui évoluent généralement sur des circuits automobiles. Le 0 à 100 km/h est abattu en moins de 3 secondes, avec une vitesse maximale de 250 km/h2. Ceux-ci sont carrossés, disposent de pneus plus larges et d'un aileron arrière.

Les châssis de karting sont dépourvus de suspension et de différentiel, le freinage est assuré par un frein à disque monté sur l'axe arrière. Certaines catégories acceptent les freins avant (petits freins à disque montés sur les moyeux avant parfois activés par une poignée au volant). L'immense majorité des karts évoluant dans les formules de promotion dispose aujourd'hui d'un démarreur électrique et d'un embrayage.

La pratique du kart se fait sur trois niveaux : en location, en loisir (avec son propre matériel) ou en compétition (courses officielles réservées aux licenciés, vitesse individuelle ou endurance).

Tirée de \url{https://fr.wikipedia.org/wiki/Karting}.

\newpage
\pagenumbering{arabic}

\section{Version intermédiaire}

Un circuit de Kart jouable pour un joueur et un créateur de circuits.

\begin{itemize}
    \item Création d'un circuit :
          \begin{itemize}
              \item Basé sur le \href{https://lj44.ch/creator/flipper}{créateur de monde du flipper} avec les modification suivantes: \begin{itemize}
                        \item Choix du tracé
                        \item Choix des dimensions
                        \item Choix du fond (background)
                        \item Choix de la musique
                        \item Choix du nombre de tours
                    \end{itemize}
          \end{itemize}
    \item Règles du jeu: \begin{itemize}
        \item Le but est de terminer le nombre de tour requis le plus rapidement possible.
    \end{itemize}
    \item Kart \begin{itemize}
        \item Collisions avec le bord de la piste
        \item Collision avec les autres karts
        \item Contrôles: \begin{itemize}
            \item Gauche
            \item Droite
            \item Accélérer
            \item Décélérer
            \item Freinage d'urgence
        \end{itemize}
    \end{itemize}
\end{itemize}

\section{Version finale}

\section{Extensions possibles}
\begin{itemize}
    \item POV du pilote du kart
    \item Minimap
    \item Possibilité de récompenses en cas de bonne performance: pièces de kart (aileron, roues, skin, ...)
    \item Kart: \begin{itemize}
        \item Possibilité de dérapage
    \end{itemize}
\end{itemize}

\section{Extensions impossibles}
\begin{itemize}
    \item IA
\end{itemize}

\end{document}